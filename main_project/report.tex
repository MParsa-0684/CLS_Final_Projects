\documentclass[12pt,a4paper]{article}
\usepackage[utf8]{inputenc}
\usepackage{geometry}
\usepackage{listings}
\usepackage{xcolor}
\usepackage{hyperref}
\geometry{margin=1in}

\title{گزارش فنی پروژه BadUSB با شبیه‌سازی HID}
\author{گروه/نام شما}
\date{\today}

% تنظیمات برای نمایش کد
\lstset{
    basicstyle=\ttfamily\footnotesize,
    keywordstyle=\color{blue},
    commentstyle=\color{green!50!black},
    stringstyle=\color{red},
    breaklines=true,
    columns=fullflexible,
    frame=single
}

\begin{document}

\maketitle

\section{مقدمه}
این پروژه شبیه‌سازی رفتار یک BadUSB با استفاده از میکروکنترلر ATtiny85 و برد توسعه Digispark در محیط Proteus است. هدف ایجاد یک HID مجازی است که بتواند داده‌ها را استخراج و بازیابی کند. پروژه شامل سه بخش اصلی است: \textbf{Firmware روی AVR}، \textbf{Bridge Python روی کامپیوتر} و \textbf{شبیه‌سازی Proteus}.

\section{اهداف پروژه}
\begin{itemize}
    \item یادگیری برنامه‌نویسی سطح پایین AVR با زبان اسمبلی و C
    \item شبیه‌سازی دستگاه‌های ورودی USB (HID) با پروتکل دست‌ساز
    \item استخراج داده‌های سیستم هدف و ارسال بازگشتی
    \item کار با حافظه RAM و EEPROM در سطح میکروکنترلر
\end{itemize}

\section{معماری سیستم}
\subsection{بخش میکروکنترلر (ATtiny85)}
کد AVR روی ATtiny85 اجرا می‌شود و UART نرم‌افزاری برای ارتباط با Bridge فراهم می‌کند. نمونه تعریف پین‌ها:
\begin{lstlisting}[language=C]
#define TX_PIN PORTB.0
#define RX_PIN PINB.1
#define BTN_PIN PINB.4
\end{lstlisting}

\textbf{فاز استخراج (Attack Mode):}  
می‌فرستد CMD/TYPE/KEY به Bridge برای اجرای دستورات در سیستم هدف و دریافت داده‌ها (مثلاً IP سیستم) و ذخیره در آرایه:
\begin{lstlisting}[language=C]
char stolen_data[32];
\end{lstlisting}

\textbf{فاز بازیابی (Playback Mode):}  
با فشردن دکمه روی برد، داده‌ها دوباره به سیستم هدف ارسال می‌شوند.

\subsection{Bridge Python}
Bridge وظیفه مدیریت دستورات و شبیه‌سازی HID را برعهده دارد. دستورات اصلی شامل:
\begin{itemize}
    \item CMD:RUN → باز کردن Run ویندوز
    \item TYPE: متن → تایپ متن
    \item KEY:ENTER → زدن Enter
    \item ACTION:GET\_CLIP → خواندن محتویات Clipboard و ارسال به MCU
\end{itemize}

\subsection{شبیه‌سازی در Proteus}
در محیط Proteus، LED‌ها (CapsLock/NumLock) برای شبیه‌سازی انتقال داده وضعیت کلیدها استفاده می‌شوند. MCU با تغییر وضعیت LED، بیت‌های داده را ارسال و دریافت می‌کند.

\section{شرح پروتکل دست‌ساز}
\begin{itemize}
    \item ارسال داده‌ها کاراکتر به کاراکتر
    \item هر دستور با '\textbackslash n' پایان می‌یابد
    \item دستورات از نوع CMD/TYPE/KEY/ACTION
    \item Bridge مسئول تشخیص دستور و اجرای عملیاتی متناسب است
\end{itemize}

\section{چالش‌های اسمبلی AVR و LED}
\begin{itemize}
    \item \textbf{Bit-banged UART:} نیاز به timing دقیق با استفاده از delay\_us
    \item \textbf{خواندن وضعیت کلید:} استفاده از polling روی پین BTN
    \item \textbf{LED Status:} برای فاز استخراج، LEDهای CapsLock/NumLock/ScrollLock نشان‌دهنده بیت‌های داده هستند
    \item \textbf{حافظه محدود:} استفاده بهینه از RAM و مدیریت آرایه‌ها
\end{itemize}

\section{مدیریت حافظه}
\begin{itemize}
    \item آرایه fixed-size 32 بایتی برای ذخیره داده‌های استخراج‌شده
    \item استفاده از توابع uart\_rx و uart\_tx برای خواندن و نوشتن بدون overflow
    \item در نسخه شبیه‌سازی، EEPROM واقعی استفاده نشده و RAM جایگزین آن شده است
\end{itemize}

\section{نمونه جریان داده}
\begin{enumerate}
    \item MCU می‌فرستد: CMD:RUN
    \item Bridge باز می‌کند Run
    \item MCU می‌فرستد: TYPE:powershell ...
    \item Bridge اجرا می‌کند و Clipboard را پر می‌کند
    \item MCU دستور ACTION:GET\_CLIP را ارسال می‌کند
    \item Bridge محتویات Clipboard را کاراکتر به کاراکتر روی UART می‌فرستد
    \item MCU داده‌ها را در stolen\_data ذخیره می‌کند
\end{enumerate}

\section{نتیجه‌گیری}
این پروژه یک شبیه‌سازی کامل BadUSB را بدون نیاز به سخت‌افزار واقعی فراهم می‌کند و مهارت‌های برنامه‌نویسی AVR، مدیریت حافظه، پروتکل‌نویسی دست‌ساز و شبیه‌سازی HID را تمرین می‌دهد.

\end{document}
