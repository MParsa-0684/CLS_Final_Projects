% Exercise Template
%A LaTeX template for typesetting exercise in Persian.
%By: Zeinab Seifoori

\documentclass[12pt]{exam}

\usepackage{setspace}
\usepackage{listings}
\usepackage{graphicx,subfigure,wrapfig}
\usepackage{multirow}
%\usepackage{multicol}

\usepackage[margin=20mm]{geometry}
\usepackage{xepersian}
\settextfont{XB Niloofar}

\newcommand{\class}{درس ساختار و زبان کامپیوتر}
\newcommand{\term}{نیم‌سال اول ۰۴-۰۵}
\newcommand{\college}{دانشکده مهندسی کامپیوتر}
\newcommand{\prof}{استاد: دکتر اسدی}

\singlespacing
\parindent 0ex

\lstset{
keywordstyle=\textbf,
identifierstyle=, 
stringstyle=\ttfamily,
commentstyle=\color{LimeGreen}, 
stringstyle=\ttfamily,
numberstyle=\footnotesize,
showstringspaces=false} 
\begin{document}

% These commands set up the running header on the top of the exam pages
\pagestyle{head}
\firstpageheader{}{}{}
\runningheader{صفحه \thepage\ از \numpages}{}{\class}
\runningheadrule
\begin{tabular}{p{.7\textwidth} l}
\multicolumn{2}{c}{\textbf{به نام خدا}}\\
\multirow{2}{*}{\includegraphics[scale=0.2] {images/logo.png}} & \\ \\
&  \textbf{\class}\\
&  \textbf{\term}\\
&  \textbf{\prof}\\ \\
 \textbf{\college} &  \\
\end{tabular}\\

\rule[1ex]{\textwidth}{.1pt}
\noindent
\textbf{ پروژه اختیاری شماره ۶ : پیاده‌سازی شبیه‌ساز ساده MIPS}
\hfill
\textbf{شماره تیم :‌ 25}

\rule[1ex]{\textwidth}{.1pt}
%\makebox[45mm]{\hrulefill}\\
\vspace{0pt}

\textbf{اطلاعات تیم دانشجویی :}
\begin{itemize}
    \item محمدپارسا محمودآبادی آرانی - 403106679
    \item سیدامیرحسین هاشمی شاهرودی - 403170271
    \item نادر احمدی - 403105674
    \item پارسا پاک - 403105822
\end{itemize}

\textbf{شرح مختصر پروژه :}\\

\textbf{هدف و ضرورت پروژه}
هدف از این پروژه، پیاده‌سازی یک شبیه‌ساز نرم‌افزاری برای پردازنده $MIPS32$ به منظور درک عمیق معماری کامپیوتر و نحوه اجرای دستورات در سطح ماشین است. در این طرح، دانشجویان با مفاهیم بنیادی نظیر واکشی دستور ($Fetch$), رمزگشایی ($Decode$) و اجرای عملیات محاسباتی و منطقی آشنا می‌شوند. ضرورت اصلی پروژه، پل زدن میان زبان‌های سطح بالا مانند $C$ یا پایتون و زبان اسمبلی است تا نحوه تعامل مستقیم پردازنده با حافظه ($Memory$) و ثبات‌ها ($Registers$) به طور عملیاتی شبیه‌سازی گردد.\\

\textbf{اجزای سخت‌افزاری شبیه‌ساز}
ساختار سخت‌افزاری این شبیه‌ساز از سه بخش حیاتی تشکیل شده است: حافظه اصلی ($RAM$) که به صورت یک آرایه برای ذخیره کدها و داده‌ها تعریف می‌شود, مجموعه ثبات‌ها ($Register\ File$) شامل ۳۲ ثبات عمومی که ثبات $reg[0]$ آن همواره مقدار صفر را نگه می‌دارد, و شمارنده برنامه ($PC$) که آدرس دستور فعلی را مدیریت می‌کند. این اجزا در قالب متغیرهای نرم‌افزاری پیاده‌سازی می‌شوند تا مدل رفتاری پردازنده را در محیط سیستم‌عامل شبیه‌سازی کنند.\\

\textbf{فرآیند شبیه‌سازی و قالب دستورات}
فرآیند شبیه‌سازی بر اساس تحلیل دقیق سه قالب اصلی دستورات $MIPS$ یعنی نوع $R-Type$ برای عملیات حسابی، نوع $I-Type$ برای دستورات شرطی و دسترسی به حافظه، و نوع $J-Type$ برای پرش‌های مستقیم استوار است. گام‌های اجرایی شامل یک چرخه تکرار شونده است که در آن ابتدا کد ماشین به صورت هگزادسیمال واکشی شده, سپس فیلدهای $Opcode$, $rs$, $rt$ و $rd$ استخراج می‌گردند و در نهایت بر اساس نوع دستور، عملیات متناظر (مانند $ADD$ یا $LW$) بر روی داده‌ها اعمال می‌شود.\\

\textbf{سناریوی عملیاتی نهایی}
در سناریوی نهایی، شبیه‌ساز یک فایل ورودی حاوی کدهای ماشین را خوانده و آن‌ها را در حافظه بارگذاری می‌کند. سیستم در هر گام، آدرس $PC$ را به‌روزرسانی کرده و واحد کنترل با بررسی $Funct$ یا $Opcode$, تصمیم مقتضی برای تغییر وضعیت ثبات‌ها یا نوشتن در حافظه را اتخاذ می‌نماید. این چرخه تا رسیدن به دستور توقف ($Halt$) ادامه می‌یابد و در پایان، وضعیت نهایی حافظه و ثبات‌ها به کاربر نمایش داده می‌شود تا صحت اجرای الگوریتم‌های اسمبلی (مانند عملیات تفریق یا مقایسه $SLT$) مورد تایید قرار گیرد.\\

\end{document}