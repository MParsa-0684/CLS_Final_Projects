% Exercise Template
%A LaTeX template for typesetting exercise in Persian.
%By: Zeinab Seifoori

\documentclass[12pt]{exam}

\usepackage{setspace}
\usepackage{listings}
\usepackage{graphicx,subfigure,wrapfig}
\usepackage{multirow}
%\usepackage{multicol}

\usepackage[margin=20mm]{geometry}
\usepackage{xepersian}
\settextfont{XB Niloofar}

\newcommand{\class}{درس ساختار و زبان کامپیوتر}
\newcommand{\term}{نیم‌سال اول ۰۴-۰۵}
\newcommand{\college}{دانشکده مهندسی کامپیوتر}
\newcommand{\prof}{استاد: دکتر اسدی}

\singlespacing
\parindent 0ex

\lstset{
keywordstyle=\textbf,
identifierstyle=, 
stringstyle=\ttfamily,
commentstyle=\color{LimeGreen}, 
stringstyle=\ttfamily,
numberstyle=\footnotesize,
showstringspaces=false} 
\begin{document}

% These commands set up the running header on the top of the exam pages
\pagestyle{head}
\firstpageheader{}{}{}
\runningheader{صفحه \thepage\ از \numpages}{}{\class}
\runningheadrule
\begin{tabular}{p{.7\textwidth} l}
\multicolumn{2}{c}{\textbf{به نام خدا}}\\
\multirow{2}{*}{\includegraphics[scale=0.2] {images/logo.png}} & \\ \\
&  \textbf{\class}\\
&  \textbf{\term}\\
&  \textbf{\prof}\\ \\
 \textbf{\college} &  \\
\end{tabular}\\

\rule[1ex]{\textwidth}{.1pt}
\noindent
\textbf{ پروژه اجباری شماره ۳ : BadUSB با شبیه‌سازی پروتکل HID}
\hfill
\textbf{شماره تیم :‌ 25}

\rule[1ex]{\textwidth}{.1pt}
%\makebox[45mm]{\hrulefill}\\
\vspace{0pt}

\textbf{اطلاعات تیم دانشجویی :}
\begin{itemize}
    \item محمدپارسا محمودآبادی آرانی - 403106679
    \item سیدامیرحسین هاشمی شاهرودی - 403170271
    \item نادر احمدی - 403105674
    \item پارسا پاک - 403105822
\end{itemize}

\textbf{شرح مختصر پروژه :}\\

\textbf{هدف و ضرورت پروژه}
هدف اصلی این پروژه، طراحی و پیاده‌سازی یک ابزار نفوذ سخت‌افزاری در سطح پایین ($Low-level$) با استفاده از پروتکل $HID$ و تراشه‌های خانواده $AVR$ است. این ابزار که تحت عنوان $BadUSB$ شناخته می‌شود، با بهره‌گیری از برد $Digispark$ مهارت‌های برنامه‌نویسی اسمبلی را برای تعامل مستقیم با پورت $USB$ و شبیه‌سازی دستگاه‌های ورودی به چالش می‌کشد. هدف نهایی، درک عمیق نحوه انتقال داده‌های حساس به صورت غیرمتعارف، بدون استفاده از کتابخانه‌های آماده و با تکیه بر منطق محض اسمبلی است.\\

\textbf{وسایل و تجهیزات مورد نیاز}
برای اجرای این پروژه، از برد توسعه $Digispark$ مبتنی بر میکروکنترلر $ATtiny85$ به عنوان هسته اصلی استفاده می‌شود. تجهیزات جانبی شامل یک عدد کلید فشاری ($Push\ Button$) برای مدیریت فاز بازیابی، مقاومت‌های لازم و در صورت نیاز، یک نمایشگر $LCD\ I2C$ برای پایش وضعیت و نمایش داده‌های استخراج شده است. در سمت نرم‌افزاری نیز، سیستم هدف می‌تواند مجهز به سیستم‌عامل ویندوز یا لینوکس باشد تا اسکریپت‌های استخراج داده را اجرا نماید.\\

\textbf{گام‌های اجرایی و فنی}
روند اجرای پروژه شامل سه گام اساسی است: ابتدا بستن مدار سخت‌افزاری و اتصال کلید و نمایشگر به پایه‌های $GPIO$ برد؛ سپس نوشتن اسکریپت‌های سمت کامپیوتر ($Bash$ یا $PowerShell$) برای استخراج اطلاعاتی نظیر رمز عبور وای‌فای؛ و در نهایت پیاده‌سازی برنامه اصلی به زبان اسمبلی $AVR$. بخش حساس فنی، طراحی پروتکل ارتباطی دست‌ساز برای خواندن وضعیت $LED$های کیبورد (مانند $Caps\ Lock$) و ذخیره‌سازی بیت‌به‌بیت داده‌ها در حافظه ماندگار $EEPROM$ میکروکنترلر است.\\

\textbf{سناریوی عملیاتی پروژه}
در سناریوی مورد انتظار، ابزار پس از اتصال به سیستم، به صورت خودکار محیط ترمینال را باز کرده و اسکریپت استخراج را اجرا می‌کند؛ سپس داده‌های به دست آمده را از طریق تغییر وضعیت چراغ‌های کیبورد به برد منتقل و ذخیره می‌کند. در مرحله بعد، نفوذگر با جدا کردن برد و اتصال مجدد آن در حالی که دکمه سخت‌افزاری را نگه داشته است، وارد فاز بازیابی می‌شود. در این فاز، برد به صورت خودکار یک نرم‌افزار ویرایشگر متن (مانند $Notepad$) را باز کرده و تمامی اطلاعات ذخیره شده در حافظه را با سرعت بالا تایپ و نمایش می‌دهد.\\

\end{document}